\chapter{Techniques d'analyses de malwares}

Topo : Introduction analyse statique / dynamique\\

Lors d'une analyse de malware, on a, en général, uniquement accès à l'exécutable quiconstitue le malware. Pour comprendre ce qu'il fait, on doit alors recourir à l'utilisationde nombreux outils et astuces.\\
Il y a deux approches à l'analyse de malwares : statique et dynamique.\\
L'analyse statique consiste à examiner le malware sans l'exécuter, tandis que l'analyse dynamique se fait pendant l'exécution du malware. Ces deux approches sont en fait complémentaires et permettent d'obtenir une vision d'ensemble sur le comportement du malware.

\section{Analyse statique}

Lecture de bytecode

\subsection{Sacnner anti-virus}

\subsection{Hash du sample et comparaison aux bases de données}

\subsection{Examen de l'exécutable (PE/ELF)}

Aide à créer un processus qui est à l'image du binaire.\\

resultat de la commande file sur un binaire linux: (ou readelf + autres)
``
ch21: setuid ELF 32-bit LSB executable, Intel 80386, version 1 (SYSV), dynamically linked, interpreter /lib/ld-linux.so.2, for GNU/Linux 2.6.9, not stripped
``

ELF (Extensible Linking Format/Executable and Linkable Format) Quantité de code et de données binaires à charger en mémoire, à quelles adresses, bibliothèques à charger, trouver le point d'entrée (si non code C), sections(libellé, adresse de début, taille et contenu), segments, endianness, 32/64 bit, ...\\
Magic-Number : 0x7F 0x45 0x4C 0x46 (7F E L F)

\begin{itemize}
\item {Symboles}
\item {Fonctions importées / exportées}
\end{itemize}

PE: structure binaire exe, bibliothèques dynamiques (dll), pilotes (sys). (0x4D 0x5A ou MZ)\\
MAC --$>$ ``mach-O'' (CAFEBABE)

\subsection{Recherche de chaînes de caractères}

Outils: strings

\subsection{Identification de packing ou d'obfuscation}

Packer --$>$ Compression du logiciel et chiffrement. Chiffrement le plus simple est un XOR où la clef est inscrite dans le fichier. Chiffrement par bloc ou entier, diffèrent au début du main.
\\
+ Protection anti-reverse ? (Techniques anti-debug (détection de breakpoint: Int 3 0xCC, détection de débogueur), checksum)

\subsection{Analyse du code assembleur}

Outils: Radare2 / IDA Pro

\section{Analyse dynamique}

\subsection{Machine virtuelle (VirtualBox / VMWare}

\begin{itemize}
\item {Mise en réseau}
\item {Snapshots}
\end{itemize}

\subsection{Tracing (strace / ProcMoon)}

\subsection{Listing des processus (ps / ProcessExplorer)}

\subsection{Simulation de réseaux (inetsim)}

\subsection{Analyse de traffic (tcpdump / wireshark)}

\subsection{Debugging (gdb / x64dgb)}

TODO: Notes sur l'automatisation des parties basiques ? (Cuckoo, Limon) <-- Pour !
