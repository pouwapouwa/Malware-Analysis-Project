\chapter{Techniques d'analyses de malwares}

Topo : Introduction analyse statique / dynamique\\

Lors d'une analyse de malware, on a, en général, uniquement accès à l'exécutable quiconstitue le malware. Pour comprendre ce qu'il fait, on doit alors recourir à l'utilisationde nombreux outils et astuces.\\
Il y a deux approches à l'analyse de malwares : statique et dynamique.\\
L'analyse statique consiste à examiner le malware sans l'exécuter, tandis que l'analysedynamique se fait pendant l'exécution du malware. Ces deux approches sont en fait complémentaires et permettent d'obtenir une vision d'ensemble sur le comportement du malware.

\section{Analyse statique}

\subsection{Sacnner anti-virus}

\subsection{Hash du sample et comparaison aux bases de données}

\subsection{Examen de l'exécutable (PE/ELF)}

\begin{itemize}
\item {Symboles}
\item {Fonctions importées / exportées}
\end{itemize}

\subsection{Recherche de chaînes de caractères}

Outils: strings

\subsection{Identification de packing ou d'obfuscation}

\subsection{Analyse du code assembleur}

Outils: Radare2 / IDA Pro

\section{Analyse dynamique}

\subsection{Machine virtuelle (VirtualBox / VMWare}

\begin{itemize}
\item {Mise en réseau}
\item {Snapshots}
\end{itemize}

\subsection{Tracing (strace / ProcMoon)}

\subsection{Listing des processus (ps / ProcessExplorer)}

\subsection{Simulation de réseaux (inetsim)}

\subsection{Analyse de traffic (tcpdump / wireshark)}

\subsection{Debugging (gdb / x64dgb)}

TODO: Notes sur l'automatisation des parties basiques ? (Cuckoo, Limon) <-- Pour !
