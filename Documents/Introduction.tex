\chapter*{Introduction}

Non sans étonnement, l'année 2016 aura été l'année du boom des ransomwares.
Avec 1,3 millions de nouveaux ransomwares détectés lors du second trimestre par
les équipes de McAfee et plus de 90\% des emails de phishing présentant des
ransomwares\footnote{\url{https://phishme.com/phishing-ransomware-threats-soared-q1-2016/}}
, la menace n'est pas à prendre à la légère.
En début d'année, un hôpital californien avait été la cible d'un ransomware,
chiffrant les données médicales de près de milliers de patients(TODO).
L'hôpital aura finalement versé 17.000\$ en bitcoins
(40 bitcoins à cette époque) afin de continuer à
pouvoir soigner ses clients (les auteurs ont accepté un prix plus faible,
la rançon originelle était de 3.4 millions de dollars)\footnote{\url
  {http://khn.org/morning-breakout/california-hospital-held-hostage-by-hackers-pays-17000-ransom-to-unlock-records/}}.
La monétisation d'informations volées ou verrouillées peut rapporter gros aux
auteurs de ces attaques.\\

Mais les ransomwares ne sont pas les seuls malwares à surveiller: ils ne
représentent qu'une partie d'une menace bien plus importante. Des spywares
(rootkits, trojans, keyloggeurs) aux downloaders et botnets,
les menaces sont variées et les cibles de plus en plus diversifiées,
avec l'essor des objets connectés (IoT) et des mobiles. Ces derniers ne
sont pas toujours sécurisés et en font des cibles de choix.\\
Il est donc important de constamment étudier ces nouvelles menaces, afin de
pouvoir protéger les systèmes d'informations et les données qu'ils contiennent.
L'analyse de malware sert typiquement trois causes:
l'obtention d'informations nécessaires à la réponse
à une intrusion, l'extraction d'indicateurs de compromission ou simplement
à la compréhension et la découverte des dernières techniques utilisées
par les fabricants de malwares.\\

Dans ce papier seront donc présentées, en première partie, les techniques
d'analyses de malware, puis deux études de cas pratiques: le botnet Ganiw et
le ransomware SageCrypt.
