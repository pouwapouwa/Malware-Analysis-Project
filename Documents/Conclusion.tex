\chapter*{Conclusion}

Les études du botnet Ganiw et du ransomware SageCrypt ont permis d'effleurer la
diversité des malwares présents aujourd'hui. Ces deux malwares sont très
différents dans leurs structures et dans les protections mises en place.\\
Ganiw est codé de manière très modulaire et ses protections contre
le reverse-engineering sont assez faible. C'est un malware facilement analysable
mais qui pourra être retrouvé sous de nombreuses formes,
avec des modules d'attaques supplémentaires.
À l'inverse, SageCrypt est bien plus protégé, avec la présence
d'un packer par exemple, mais son code et sa structure sont plus ou moins connus,
car dérivés d'un autre ransomware, CryLocker.\\

À la suite de cette étude, il est facile de remarquer que du temps peut être
économisé en automatisant certaines parties de l'étude, qui plus est si
l'élément a inspecté ou que ses mécanismes d'infection et de propagation
sont déjà, en partie connus, et si de nombreux éléments sont à étudier.
Parmi les outils existants sont disponibles les sandboxs
\textit{Limon}\footnote{\url{https://github.com/monnappa22/Limon}} et
\textit{Cuckoo}\footnote{\url{https://github.com/cuckoosandbox/cuckoo}},
très utiles pour fournir rapidement de nombreuses informations sur l'élément
analysé et de manière sécurisé, sans compromettre son environnement de travail.\\

Mais ceci ne permet pas de classifier les malwares et nécessite encore
une grande partie de travail manuel. Pour cela, il faudrait pouvoir les
discriminer en fonction de leur comportement, de leurs appels
à diverses librairies, \ldots Ceci pourrait être fait en se basant sur
le graphe du flot d'exécution (CFG: Control Flow Graph), mais devra être fiable
et les analyses de malware ne pourront se passer d'actions humaines pour
identifier de nouvelles formes de menaces.
