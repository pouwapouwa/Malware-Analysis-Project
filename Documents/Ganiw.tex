\chapter{Ganiw}

\section{Introduction}

TODO: Introduction sur les botnets

Le \textit{sample} porté à l'étude porte le doux nom de \textbf{Ganiw} mais est
aussi connu sous les nom de \textbf{BillGates} ou de combinaison comme
\textbf{LINUX.BACKDOOR.GATES}.\\
TODO:md5/sha1\\
Celui-ci semble avoir été étudié pour la première fois en février 2014
sur un site russe: \textit{\href{https://habrahabr.ru/post/213973/}
{Studying the BillGates Linux Botnet}}\footnotemark,
mais continue tout de même à être étudié et à être actif.
En effet, comme il sera expliqué plus amplement par la suite, celui-ci présente
des propriétés particulières.\\

Ce malware est présent uniquement sur les systèmes Linux.
Ce qui n'est pas anodin et sans lien quand à son utilisation: le
développement toujours plus important des objets connectés dans l'IoT
en fait une cible de choix. Linux ou les systèmes basés sur son noyau
(comme avec le projet Yocto\footnotemark) sont donc
des RTOS parfaits (TODO à reformuler) pour de petits appareils
car légers et open-sources et y ont le monopole.
De plus, ces objets connectés sont, en règle générale, peu protégés
(absence de firewall, mots de passe faibles,\ldots)
et non surveillés ou peu maintenus.
Ce sont donc des cibles de choix pour un malware de type \textit{BotNet},
comme \textbf{Ganiw}, pour mener des attaques réseaux de types DDOS par exemple.\\

La suite de cette étude va se faire sous 2 formes: une partie par
analyse statique du binaire et une autre par analyse dynamique,
 permettant de confirmer et approfondir les données recueillies précédemment.

\footnotetext{Traduit depuis le russe}
\footnotetext{\url{https://www.yoctoproject.org/}}

\section{Étude -- Analyse statique}

\subsection{Outils d'analyse}

TODO: description outils et buts d'analyse statique

Description des différents modules avec captures d'écrans ou diagrammes\\
Fonctionnalités autres (hook dans init.d)\\

\subsection{Étude -- Main}

Développé en C++, not stripped, "technique anti-debug" (parent !=gdb ← wow)
, fonctionnement modulaire\\

3 modes de fonctionnement --> Screenshot r2\\

Fonctionnement dépend depuis où est lancé le malware.

\subsection{Étude -- Beikong}

%Depuis le main, HGrd9 écrit DbSecuritySpt (g\_strSN), selinux (g\_strBSDN), getty (g\_strBDG), /tmp/moni.lod (g__strML), (g__strGL) avec 'nkfsd8' qui écrit tmp, 'Kusdf9' qui écrit moni, tRd76 qui écrit lod. KDS87y inutil !! 0sdku6 '/tmp/moni.lod'
%\\

\subsection{Étude -- Backdoor}

\subsection{Étude -- Autres modules}

\section{Étude -- Analyse dynamique}

inetsim gdb strace wireshark/tcpdump ++ mise en place vm/dns server/...

Mise en place d'un environnement de travail $+$ outils utilisés (lemon ?)\\
→ crontab trouvé dans l'analyse statique --> fonctionnement fichier rc.

\section{Conclusion}
